\documentclass[a4paper,11pt]{article}
\usepackage[T1]{fontenc}

\title{TI2206 Software Engineering: Bubbleshooter}
\author{Gerlof Fokkema 4257286 \\
	Owen Huang 4317459 \\
	Adam Iqbal 4293568 \\
	Nando Kartoredjo 4271378 \\
	Skip Lentz 4334051 \\
}

\begin{document}
\maketitle

\newpage
\section*{Functional}
\begin{itemize}
  \item User will be presented a menu after starting the application
  \begin{itemize}
    \item user can start a game
    \item user can exit the application
  \end{itemize}
  \item User can play a game (single player mode)
  \begin{itemize}
    \item user can pause the game
    \item user can exit the game (go back to main menu)
    \item the player can shoot bubbles
    \item the player can control the cannon
    \item when three bubbles of the same color are adjacent these will disappear
    \item bubbles stick to each other when the condition above does not hold
    \item the player can form groups of the same color to score points
    \item bubbles that aren't connected to the ceiling will fall to the floor
    \item when the bubbles reach the floor the player loses and the game exits
  \end{itemize}
\end{itemize}

\newpage
\section*{Non-functional}
\begin{itemize}
  \item A simple version should be finished within 1 week (14/09/14).
  \item The development team consists of five group members.
  \item The game must be written in Java using the following supporting tools:
  \begin{itemize}
    \item maven
    \item jUnit
    \item git
    \item libgdx
  \end{itemize}

  \item The development process will be using SCRUM
  \begin{itemize}
    \item ScrumDo
  \end{itemize}

  \item Meetings
  \begin{itemize}
    \item Friday 20:00 - 05/09/2014 (Daily sprint)
    \item Monday 20:00 - 08/09/2014 (Daily sprint)
    \item Tuesday 9:45 - 09/09/2014 (Sprint planning)
    \item Wednesday 20:00 - 10/09/2014 (Daily sprint)
    \item Friday 20:00 - 12/09/2014 (Sprint review)
    \item Tuesday 9:45 - 16/09/2014 (Sprint retrospective)
  \end{itemize}

  \item The game must support the following OS:
  \begin{itemize}
    \item Microsoft Windows
    \item Linux
    \item OS X
  \end{itemize}
\end{itemize}
\end{document}
