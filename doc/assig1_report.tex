\documentclass[a4paper,11pt]{article}
\usepackage[T1]{fontenc}
\usepackage{hyperref}
\usepackage{titlesec}

%\titleformat{\subsection}
%[display]
%{\large\bfseries}
%{Exercise \thesubsection}{1pt}{}

% Star to kill indentation of following par.
%\titlespacing*{\subsection}
%{0pt}{5pt}{-15pt}


\titleformat{\section}
[runin]
{\LARGE\bfseries}
{Exercise \thesection: \rightmark}{1pt}{}

\titleformat{\subsection}
{\large\bfseries}
{Sub-exercise \thesubsection}{1pt}{}

% Star to kill indentation of following par.
\titlespacing*{\subsection}
{0pt}{5pt}{0pt}

\title{TI2206 Software Engineering \\ Bubble Shooter report: Assignment 1 \\ EEMCS/EWI}
\author{Gerlof Fokkema 4257286 \\
	Owen Huang 4317459 \\
	Adam Iqbal 4293568 \\
	Nando Kartoredjo 4271378 \\
	Skip Lentz 4334051 \\
}

\begin{document}
\maketitle
\thispagestyle{empty}

\newpage
\setcounter{page}{1}

%###################################### EXERCISE 1 ###################################### 
\section{The core}

%~~~~~~~~~~~~ SUB-EXERCISE 1.1 ~~~~~~~~~~~~
\subsection{}
The two \underline{main} classes that have already been implemented, described in terms of responsibility and collaborations are as follows:
\begin{enumerate}
  \item First, one of the most important classes is the \textit{Board} class. This class describes the entire playing-field. It can also place and remove bubbles, and it can return the clusters of bubbles of the same color. It collaborates with the \textit{BubbleShooterScreen} class and the \textit{Bubble} class.
  \item Secondly, the \textit{BubbleShooterScreen} class should also be considered one of the main classes. It is responsible for drawing all the game elements (bubbles, cannon and background) onto the screen, for the game logic and for applying the game's rules. Additionally it takes care of the user input and acts accordingly. It collaborates with the \textit{Board} class and the \textit{Cannon} class.
\end{enumerate}

%~~~~~~~~~~~~ SUB-EXERCISE 1.2 ~~~~~~~~~~~~
\subsection{}
The reason why the other classes are not considered main classes, is because of the fact that they carry far less responsibility. Those classes do not collaborate with that many classes in comparison to the main classes. For example, the \textit{Bubble} class does not "know" about any other classes, and its only responsibility is basically to exist.\\\\
\noindent
Looking at the current structure, some changes to the main classes are in order.

First up is the \textit{BubbleShooterScreen} class. The name of the class does not correspond to its actual responsibility within the program. \textit{BubbleShooterScreen} should only take care of the rendering of the game, and the User Interface. Therefore, the game logic and user input code should both go to their respective classes.

Then, the \textit{Board} class' grid implementation and behaviour should also be encapsulated. This way, we can always substitute the hexagonal grid for a square grid, for example.

%###################################### EXERCISE 2 ###################################### 
\section{Theory in practice}
%###################################### EXERCISE 3 ###################################### 
\section{Play with your friends}
\end{document}
